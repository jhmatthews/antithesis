\chapter{Monte Carlo Radiative Transfer and Ionization}

In the previous chapters I have given, in fairly broad brush strokes,
an introduction to the field and som relevant background relating to accretion 
disks and their associated outflows. Now it proves useful
to discuss some of the specific {\sl methods} one might be able to utilise 
in order to try and answer some of the questions raised in the previous sections.
In particular, I will discuss radiative transfer techniques and 
their potential applications.

\section{Fundamentals of Radiative Transfer}

\py\ is a Monte Carlo  ionization and radiative transfer code which
uses the Sobolev approximation to treat line transfer 
\citep[e.g.][]{sobolev1957,sobolev1960,rybickihummer1978}. 
The code has already been described extensively by LK02, SDL05 and 
Higginbottom et al. (2013; hereafter H13), so here we provide only a brief summary of its operation, 
focusing particularly on new aspects of our implementation of macro-atoms into the code.

\section{\sc{python}: A Monte Carlo Ionization and Radiative Transfer Code}

\py\ operates in two distinct stages. First, the ionization state,
level populations and temperature structure are calculated. This is
done iteratively, by propagating several populations of Monte Carlo energy quanta (`photons')
through a model wind. The geometric and kinemetic properties of the
outflow are specified on a pre-defined spatial grid. In each of these
iterations (`ionization cycles'), the code records estimators that 
characterize the radiation field in each grid cell. At the end 
of each ionization cycle, a new electron temperature is calculated
that more closely balances heating and cooling in the 
plasma. The radiative estimators and updated electron
temperature are then used to revise the ionization state of the wind,
and a new ionization cycle is started. The process is repeated until
heating and cooling are balanced throughout the wind. 

This converged model is then used as the basis for the second set of
iterations (`spectral cycles'). In these, the emergent spectrum over
the desired spectral range is synthesized by tracking populations of
energy packets through the wind and computing the emergent spectra at
a number of user-specified viewing angles.  

\py\ is designed to operate in a number of different
regimes, both in terms of the scale of the system and in terms of the
characteristics of the underlying radiation field.
It was originally developed by LK02 in order to model the UV spectra
of CVs with a simple biconical disk wind model. SDL05
\nocite{simmacro2005} used the code to model Brackett
and Pfund line profiles of H in young-stellar objects (YSOs). As part
of this effort, they implemented a `macro-atom' mode (see below) in
order to correctly treat H recombination lines with
\py. Finally, H13 used \py\ to model broad absorption line (BAL) QSOs. For
this application, an improved treatment of ionization was implemented,
so that the code is now capable of dealing with arbitrary
photo-ionizing SEDs, including non-thermal and multi-component ones. 

\section{Macro-atoms}

Lucy (2002, 2003\nocite{lucy2002, lucy2003}; hereafter L02, L03) 
has shown that it is possible to calculate the emissivity of a gas in
statistical equilibrium accurately by quantising matter into
`macro-atoms', and radiant and kinetic energy into indivisible energy
packets (r- and k- packets, respectively). His macro-atom scheme
allows for all possible transition paths from a given level and
provides a full non-local thermodynamic equilibrium (NLTE) solution
for the level populations based on Monte Carlo estimators. The macro-atom
technique has already been used to model Wolf-Rayet star
winds \citep{sim2004}, AGN disk winds \citep{simlong2008, tatum2012},
supernovae \citep{kromersim2009, kerzendorfsim} and YSOs (SDL05). A full 
description of the approach can be found in L02 and L03. 

The fundamental approach here requires somewhat of a philosophical shift.
Normally MCRT is described in the most intuitive way- that is, we imagine
real photons striking atoms and scattering, or photoionizing 
and depositing energy in a plasma. With Lucy's scheme one should instead 
reimagine the MC quanta as a packets of quantised energy flow, and the scheme as a 
{\em statistical} one. The amount of time a given energy quanta spends in a specific atomic
level or thermal pool is then somewhat analogous to the absolute energy 
contained therein.

Following L02, let us consider an atomic species interacting with a radiation field.
If the quantity $\epsilon_j$ represents the ionization plus excitation energy of 
a level i then the rates at which the level $j$ absorbs and emits radiant energy 
are given by

\begin{eqnarray}
\dot{A}_R^j =  \epsilon_j R_{jl} && \dot{E}_R^j =  \epsilon_j R_{jl}
\end{eqnarray}