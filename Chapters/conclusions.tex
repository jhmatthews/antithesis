\chapter{Conclusions}

\epigraph{``...and the credits rave as the critics roll.''}
{{\sl Mike Vennart, Silent/Transparent}}

I began this thesis with the fundamental tenet that accreting systems
and their associated outflows are astrophysically important, but that
much of the diverse phenomenology associated with such systems, as well
as the underlying {\em physics}, is not well understood.  
Having attempted to address some of the issues raised in the 
introductory chapters, I will provide some concluding remarks. 
First, I will summarise my findings, before 
commenting on how future research can unveil the true nature of 
accretion discs and their winds.

A large portion of the time of this PhD has been spent maintaining,
testing and developing the MCRT and ionization code, \py. The first 
step in this thesis was 



In the first study I demonstrated that accretion disc winds
can have a profound impact on the optical spectra of CVs -- a region
of the spectrum often assumed to have nothing to do with outflows.

In chapter 5, I applied similar techniques to the question of quasar 
unification.

\section{Suggestions for Future Work}

\subsection{CVs as Accretion and Outflow Laboratories}

\subsection{Improving the Treatment of Clumping}

\subsection{Expanding the Capabilities of \py}

\subsection{Obtaining Reliable Orientation Indicators}

\section{Closing Remarks}












